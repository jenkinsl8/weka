% Version: $Revision$

TODO:

\section{Mailing list}
The WEKA Mailing list can be found here:

\begin{itemize}
\item \verb=http://list.scms.waikato.ac.nz/mailman/listinfo/wekalist=\\
for subscribing/unsubscribing the list
\item \verb=https://list.scms.waikato.ac.nz/pipermail/wekalist/=\\
(Mirrors: \verb=http://news.gmane.org/gmane.comp.ai.weka=, \verb=http://www.nabble.com/WEKA-f435.html=)\\
for searching previous posted messages
\end{itemize}

\noindent Before posting, please read the Mailing List Etiquette\\
(\verb=http://www.cs.waikato.ac.nz/~ml/weka/mailinglist_etiquette.html=).


%$http://weka.sourceforge.net/wekadoc/index.php/en:Weka_Mailing_List_(3.5.6)$

\section{Troubleshooting}
Here are a few of things that are useful to know when you are having
trouble installing or running Weka successfullyon your machine.

\textbf{NB} these java commands refer to ones executed in a shell
(bash, command prompt, etc.) and \textbf{NOT} to commands executed in the
SimpleCLI.

\subsection{Weka download problems}
When you \textbf{download Weka}, make sure that the resulting file size is the
same as on our webpage. Otherwise things won't work
properly. Apparently some web browsers have trouble downloading Weka.

\subsection{OutOfMemoryException}
Most Java virtual machines only allocate a certain maximum amount of
memory to run Java programs. Usually this is much less than the amount
of RAM in your computer. However, you can extend the memory available
for the virtual machine by setting appropriate options. With Sun's
JDK, for example, you can go\\

\verb=java -Xmx100m ...=\\

\noindent to set the maximum Java heap size to 100MB. For more
information about these options see
\verb=http://java.sun.com/docs/hotspot/VMOptions.html=.

\subsubsection{Windows}

\textbf{Book version}\\
You have to modify the JVM invocation
in the \verb=RunWeka.bat= batch file in your installation directory.\\

\noindent \textbf{Developer version}
\begin{itemize}
\item up to \textbf{Weka 3.5.2}\\
just like the book version.
\item \textbf{Weka 3.5.3}\\
You have to modify the link in the Windows Start menu, if you're starting the console-less Weka (only the link with console in its name executes the \verb=RunWeka.bat= batch file)
\item \textbf{Weka 3.5.4} and higher
Due to the new launching scheme, you no longer modify the batch file, but the \verb=RunWeka.ini= file. In that particular file, you'll have to change the \textbf{maxheap} placeholder. See section~\ref{RunWeka.ini}.
\end{itemize}

\subsection{Mac OSX}
In your Weka installation directory (\verb=weka-3-x-y.app=) locate the \verb=Contents= subdirectory and edit the
\verb=Info.plist= file. Near the bottom of the file you should see some text like:

\begin{verbatim}
<key>VMOptions</key>
<string>-Xmx256M</string>
\end{verbatim}

\noindent Alter the \verb=256M= to something higher.

\subsection{StackOverflowError}
Try increasing the stack of your virtual machine. With Sun's JDK you
can use this command to increase the stacksize:\\

\verb=java -Xss512k ...=\\

\noindent to set the maximum Java stack size to 512KB. If still not
sufficient, slowly increase it.

\subsection{just-in-time (JIT) compiler}
For maximum enjoyment, use a virtual machine that incorporates a
\textbf{just-in-time compiler}. This can speed things up quite
significantly. Note also that there can be large differences in
execution time between different virtual machines.

\subsection{CSV file conversion}
Either load the CSV file in the Explorer or use the CVS converter on
the commandline as follows:\\

\verb=java weka.core.converters.CSVLoader filename.csv > filename.arff=\\

\subsection{ARFF file doesn't load}
One way to figure out why ARFF files are failing to load is to give
them to the Instances class. At the command line type the following:\\

\verb=java weka.core.Instances filename.arff=\\

\noindent where you substitute 'filename' for the actual name of your
file. This should return an error if there is a problem reading the
file, or show some statistics if the file is ok. The error message you
get should give some indication of what is wrong.

\subsection{Spaces in labels of ARFF files}
A common problem people have with ARFF files is that labels can only
have spaces if they are enclosed in single quotes, i.e. a label such
as:\\

\verb=some value=\\

\noindent should be written either 'some value' or some\_value in the file.

\subsection{CLASSPATH problems}
Having problems getting Weka to run from a DOS/UNIX command prompt?
Getting \verb=java.lang.NoClassDefFoundError= exceptions? Most likely your
\textbf{CLASSPATH} environment variable is not set correctly - it needs to
point to the Weka.jar file that you downloaded with Weka (or the
parent of the Weka directory if you have extracted the jar). Under DOS
this can be achieved with:\\

\verb^set CLASSPATH=c:\weka-3-4\weka.jar;%CLASSPATH%^\\

\noindent Under UNIX/Linux something like:\\

\verb^export CLASSPATH=/home/weka/weka.jar:$CLASSPATH^\\

\noindent An easy way to get avoid setting the variable this is to
specify the CLASSPATH when calling Java. For example, if the jar file
is located at c:$\backslash$weka-3-4$\backslash$weka.jar you can use:\\

\verb^java -cp c:\weka-3-4\weka.jar weka.classifiers... etc.^\\

\noindent See also Section~\ref{CLASSPATH}.

\subsection{Instance ID}
People often want to \textbf{tag} their \textbf{instances with
  identifiers}, so they can keep track of them and the predictions
made on them.

\subsubsection{Adding the ID}
A new ID attribute is added real easy: one only needs to run the \verb=AddID=
filter over the dataset and it's done. Here's an example (at a
DOS/Unix command prompt):\\

\begin{verbatim}
java weka.filters.unsupervised.attribute.AddID
  -i data_without_id.arff
  -o data_with_id.arff
\end{verbatim}

\noindent (all on a single line).

\textbf{Note}: the AddID filter adds a numeric attribute, not a String
attribute to the dataset. If you want to remove this ID attribute for
the classifier in a \verb=FilteredClassifier= environment again, use the
\verb=Remove= filter instead of the \verb=RemoveType= filter (same package).

\subsubsection{Removing the ID}
If you run from the command line you can use the \verb=-p= option to output
predictions plus any other attributes you are interested in. So it is
possible to have a string attribute in your data that acts as an
identifier. A problem is that most classifiers don't like String
attributes, but you can get around this by using the \verb=RemoveType= (this
removes \textit{String} attributes by default).

Here's an example. Lets say you have a training file named \verb=train.arff=,
a testing file named \verb=test.arff=, and they have an identifier String
attribute as their 5th attribute. You can get the predictions from \verb=J48=
along with the identifier strings by issuing the following command (at
a DOS/Unix command prompt):

\begin{verbatim}
java weka.classifiers.meta.FilteredClassifier
  -F weka.filters.unsupervised.attribute.RemoveType
  -W weka.classifiers.trees.J48
  -t train.arff -T test.arff -p 5
\end{verbatim}

\noindent (all on a single line).

If you want, you can redirect the output to a file by adding
``\verb=> output.txt='' to the end of the line.

In the Explorer GUI you could try a similar trick of using the String
attribute identifiers here as well. Choose the \verb=FilteredClassifier=,
with \verb=RemoveType= as the filter, and whatever classifier you
prefer. When you visualize the results you will need click through
each instance to see the identifier listed for each.

\subsection{Visualization}
Access to \textbf{visualization} from the ClassifierPanel, ClusterPanel and
AttributeSelection panel is available from a popup menu. Click the
right mouse button over an entry in the Result list to bring up the
menu. You will be presented with options for viewing or saving the
text output and---depending on the scheme---further options for
visualizing errors, clusters, trees etc.

\subsection{Memory consumption and Garbage collector}
There is the ability to print \textbf{how much memory is available} in the
Explorer and Experimenter and to run the garbage collector. Just right
click over the Status area in the Explorer/Experimenter.

\subsection{GUIChooser starts but not Experimenter or Explorer}
\label{GUIChooserButNotExplorer}
The GUIChooser starts, but Explorer and Experimenter don't start and
output an Exception like this in the terminal:

\begin{verbatim}
/usr/share/themes/Mist/gtk-2.0/gtkrc:48: Engine "mist" is unsupported, ignoring
---Registering Weka Editors---
java.lang.NullPointerException
       at weka.gui.explorer.PreprocessPanel.addPropertyChangeListener(PreprocessPanel.java:519)
       at javax.swing.plaf.synth.SynthPanelUI.installListeners(SynthPanelUI.java:49)
       at javax.swing.plaf.synth.SynthPanelUI.installUI(SynthPanelUI.java:38)
       at javax.swing.JComponent.setUI(JComponent.java:652)
       at javax.swing.JPanel.setUI(JPanel.java:131) 
       ...
\end{verbatim}

\noindent This behavior happens only under Java 1.5 and Gnome/Linux,
KDE doesn't produce this error. The reason for this is, that Weka
tries to look more ``native'' and therefore sets a platform-specific
Swing theme. Unfortunately, this doesn't seem to be working correctly
in Java 1.5 together with Gnome. A workaround for this is to set the
cross-platform \textbf{Metal} theme.

In order to use another theme one only has to create the following
properties file in ones home directory:\\

\verb=LookAndFeel.props=\\

\noindent With this content:\\

\verb^Theme=javax.swing.plaf.metal.MetalLookAndFeel^\\

\subsection{KnowledgeFlow toolbars are empty}
In the terminal, you will most likely see this output as well:\\

\verb=Failed to instantiate: weka.gui.beans.Loader=\\

\noindent This behavior can happen under Gnome with Java 1.5, see
Section~\ref{GUIChooserButNotExplorer} for a solution.

\subsection{Links}
\begin{itemize}
\item Java VM options (\verb=http://java.sun.com/docs/hotspot/VMOptions.html=)
\end{itemize}


%$http://weka.sourceforge.net/wekadoc/index.php/en:Troubleshooting_(3.5.6)$
