% Version: $Revision$

A common query we get from our users is how to open a Windows database in the Weka Explorer. This page is intended as a guide to help you achieve this. It is a complicated process and we cannot guarantee that it will work for you. The process described makes use of the JDBC-ODBC bridge that is part of Sun's JRE 1.3 (and higher).

The following instructions are for Windows 2000. Under other Windows versions there may be slight differences.

\section*{Step 1: Create a User DSN}
\begin{enumerate}
	\item Go to the \textbf{Control Panel}
	\item Choose \textbf{Adminstrative Tools}
	\item Choose \textbf{Data Sources (ODBC)}
	\item At the \textbf{User DSN} tab, choose \textbf{Add...}
	\item Choose database
	\begin{itemize}
		\item Microsoft Access
		\begin{enumerate}
			\item Note: Make sure your database is not open in another application before following the steps below.
			\item Choose the \textbf{Microsoft Access} driver and click \textbf{Finish}
			\item Give the source a name by typing it into the \textbf{Data Source Name} field
			\item In the \textbf{Database} section, choose \textbf{Select...}
			\item Browse to find your database file, select it and click \textbf{OK}
			\item Click \textbf{OK} to finalize your DSN
		\end{enumerate}

		\item Microsoft SQL Server 2000 (Desktop Engine)
		\begin{enumerate}
			\item Choose the \textbf{SQL Server} driver and click \textbf{Finish}
			\item Give the source a name by typing it into the \textbf{Name} field
			\item Add a description for this source in the \textbf{Description} field
			\item Select the server you're connecting to from the \textbf{Server} combobox
			\item For the verification of the authenticity of the login ID choose \textbf{With SQL Server...}
			\item Check \textbf{Connect to SQL Server to obtain default settings...} and supply the user ID and password with which you installed the Desktop Engine
			\item Just click on \textbf{Next} until it changes into \textbf{Finish} and click this, too
			\item For testing purposes, click on \textbf{Test Data Source...} - the result should be \textit{TESTS COMPLETED SUCCESSFULLY!}
			\item Click on \textbf{OK}
		\end{enumerate}

		\item MySQL
		\begin{enumerate}
			\item Choose the \textbf{MySQL ODBC} driver and click \textbf{Finish}
			\item Give the source a name by typing it into the \textbf{Data Source Name} field
			\item Add a description for this source in the \textbf{Description} field
			\item Specify the server you're connecting to in \textbf{Server}
			\item Fill in the user to use for connecting to the database in the \textbf{User} field, the same for the password
			\item Choose the database for this DSN from the \textbf{Database} combobox
			\item Click on \textbf{OK}
		\end{enumerate}
	\end{itemize}
	\item Your DSN should now be listed in the \textbf{User Data Sources} list
\end{enumerate}

\section*{Step 2: Set up the DatabaseUtils.props file}
You will need to create a file called DatabaseUtils.props. This file already exists under the path \texttt{weka/experiment/} in the \texttt{weka.jar} file that is part of the Weka download. In this directory you will also find a sample file for ODBC connectivity, called \texttt{DatabaseUtils.props.odbc}. You can use that as basis, since it already contains default values specific to ODBC access.

This file needs to be recognized when the Explorer starts. You can achieve this by making sure it is in the working directory, or by replacing the version the already exists in the \texttt{weka/experiment} directory. A way of achieving the second alternative would be to extract the contents of the \texttt{weka.jar}, and setting your CLASSPATH to point to the directory where \texttt{weka} resides rather that the \texttt{.jar} file.

The file is a text file that needs to contain the following lines:

\begin{verbatim}
 jdbcDriver=sun.jdbc.odbc.JdbcOdbcDriver
 jdbcURL=jdbc:odbc:dbname
\end{verbatim}

\noindent where \textit{dbname} is the name you gave the user DSN. (This can also be changed once the Explorer is running.)

\newpage
\section*{Step 3: Open the database}
\begin{enumerate}
	\item Start up the Weka Explorer. If you want to be sure that the \texttt{DatabaseUtils.props} file is in the current path, you can open a command prompt window, change to the directory where the \texttt{DatabaseUtils.props} file is located, make sure your CLASSPATH environment variable is set correctly (or set it with the \texttt{-cp} option to java) and launch the Explorer with the following command:
	\begin{verbatim}
	  java weka.gui.explorer.Explorer 
	\end{verbatim}
	\item Choose \textbf{Open DB...}
	\item Edit the \textbf{query} field to read "select * from \textit{tablename}" where \textit{tablename} is the name of the database table you want to read, or you could put a more complicated SQL query here instead.
	\item The \textbf{databaseURL} should read "jdbc:odbc:\textit{dbname}" where \textit{dbname} is the name you gave the user DSN.
	\item Click OK
\end{enumerate}

\noindent At this point the data should be read from the database.
