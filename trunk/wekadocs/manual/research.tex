%
%    This program is free software; you can redistribute it and/or modify
%    it under the terms of the GNU General Public License as published by
%    the Free Software Foundation; either version 2 of the License, or
%    (at your option) any later version.
%
%    This program is distributed in the hope that it will be useful,
%    but WITHOUT ANY WARRANTY; without even the implied warranty of
%    MERCHANTABILITY or FITNESS FOR A PARTICULAR PURPOSE.  See the
%    GNU General Public License for more details.
%
%    You should have received a copy of the GNU General Public License
%    along with this program; if not, write to the Free Software
%    Foundation, Inc., 675 Mass Ave, Cambridge, MA 02139, USA.
%

% Version: $Revision$

\section{Citing Weka}
If you want to refer to Weka in a publication, please cite following SIGKDD
Explorations\footnote{\url{
http://www.kdd.org/explorations/issues/11-1-2009-07/p2V11n1.pdf}} paper. The
full citation is:

\begin{quote}
Mark Hall, Eibe Frank, Geoffrey Holmes, Bernhard Pfahringer, Peter Reutemann,
Ian H. Witten (2009); \textit{The WEKA Data Mining Software: An Update}; SIGKDD
Explorations, Volume 11, Issue 1. 
\end{quote}

\section{Paper references}
Due to the introduction of the \texttt{weka.core.TechnicalInformationHandler} interface it is now easy to extract all the paper references via \texttt{weka.core.ClassDiscovery} and \texttt{weka.core.TechnicalInformation}.

The script listed at the end, extracts all the paper references from Weka based on a given jar file and dumps it to stdout. One can either generate simple plain text output (option \texttt{-p}) or BibTeX compliant one (option \texttt{-b}).

Typical use (after an \texttt{ant exejar}) for BibTeX:

\begin{verbatim}
  get_wekatechinfo.sh -d ../ -w ../dist/weka.jar -b > ../tech.txt
\end{verbatim}

\noindent (command is issued from the same directory the Weka \texttt{build.xml} is located in)

\newpage
\noindent Bash shell script \texttt{get\_wekatechinfo.sh}
{\scriptsize
% Version: $Revision$

\begin{verbatim}
#!/bin/bash
#
# This script prints the information stored in TechnicalInformationHandlers
# to stdout.
#
# FracPete, $Revision$

# the usage of this script
function usage()
{
   echo
   echo "${0##*/} -d <dir> [-w <jar>] [-p|-b] [-h]"
   echo
   echo "Prints the information stored in TechnicalInformationHandlers to stdout."
   echo
   echo " -h   this help"
   echo " -d   <dir>"
   echo "      the directory to look for packages, must be the one just above"
   echo "      the 'weka' package, default: $DIR"
   echo " -w   <jar>"
   echo "      the weka jar to use, if not in CLASSPATH"
   echo " -p   prints the information in plaintext format"
   echo " -b   prints the information in BibTeX format"
   echo
}

# generates a filename out of the classname TMP and returns it in TMP
# uses the directory in DIR
function class_to_filename()
{
  TMP=$DIR"/"`echo $TMP | sed s/"\."/"\/"/g`".java"
}

# variables
DIR="."
PLAINTEXT="no"
BIBTEX="no"
WEKA=""
TECHINFOHANDLER="weka.core.TechnicalInformationHandler"
TECHINFO="weka.core.TechnicalInformation"
CLASSDISCOVERY="weka.core.ClassDiscovery"

# interprete parameters
while getopts ":hpbw:d:" flag
do
   case $flag in
      p) PLAINTEXT="yes"
         ;;
      b) BIBTEX="yes"
         ;;
      d) DIR=$OPTARG
         ;;
      w) WEKA=$OPTARG
         ;;
      h) usage
         exit 0
         ;;
      *) usage
         exit 1
         ;;
   esac
done

# either plaintext or bibtex
if [ "$PLAINTEXT" = "$BIBTEX" ]
then
   echo
   echo "ERROR: either -p or -b has to be given!"
   echo
   usage
   exit 2
fi

# do we have everything?
if [ "$DIR" = "" ] || [ ! -d "$DIR" ]
then
   echo
   echo "ERROR: no directory or non-existing one provided!"
   echo
   usage
   exit 3
fi

# generate Java call
if [ "$WEKA" = "" ]
then
  JAVA="java"
else
  JAVA="java -classpath $WEKA"
fi
if [ "$PLAINTEXT" = "yes" ]
then
  CMD="$JAVA $TECHINFO -plaintext"
elif [ "$BIBTEX" = "yes" ]
then
  CMD="$JAVA $TECHINFO -bibtex"
fi

# find packages
TMP=`find $DIR -mindepth 1 -type d | grep -v CVS | sed s/".*weka"/"weka"/g | sed s/"\/"/./g`
PACKAGES=`echo $TMP | sed s/" "/,/g`

# get technicalinformationhandlers
TECHINFOHANDLERS=`$JAVA weka.core.ClassDiscovery $TECHINFOHANDLER $PACKAGES | grep "\. weka" | sed s/".*weka"/weka/g`

# output information
echo
for i in $TECHINFOHANDLERS
do
  TMP=$i;class_to_filename

  # exclude internal classes
  if [ ! -f $TMP ]
  then
    continue
  fi

  $CMD -W $i
  echo
done
\end{verbatim}

} 
