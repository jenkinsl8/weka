% Version: $Revision$

Using the graphical tools, like the Explorer, or the command-line is in most
cases sufficient for the normal user. But WEKA's clearly defined API
(``application programming interface'') makes is very easy to ``embed'' it in
another projects. This chapter covers some basics of how to achieve the
following common tasks from source code:
\begin{tight_itemize}
	\item Option handling
	\item Loading data
	\item Filtering
	\item Classifying
	\item Clustering
	\item Selecting attributes
\end{tight_itemize}
Even though most of the examples are for the Linux platform, using forward
slashes in the paths and file names, they do work on the MS Windows platform as
well. To make the examples work under MS Windows, one only needs to adapt the
paths, changing the forward slashes to backslashes and adding a drive letter
where appropriate. \\

\noindent \textbf{Note} \\
\noindent WEKA is released under the GNU General Public License version
2\footnote{\url{http://www.gnu.org/licenses/gpl-2.0.html}{}} (GPLv2), i.e., that
derived
code or code that uses WEKA needs to be released under the GPLv2 as well. If
you are just using WEKA for a personal project that does not get released
publicly then you are not affected. But as soon as you make your project
publicly available (e.g., for download), then you need to make your source code
available under the GLPv2 as well, alongside the binaries.

\newpage

%%%%%%%%%%%%%%%%%%%
% Option handling %
%%%%%%%%%%%%%%%%%%%
\section{Option handling}
Configuring an object, e.g., a classifier, can either be done using the
appropriate get/set-methods for the property that you wish to change, like the
Explorer does. Or, if the class implements the \texttt{weka.core.OptionHandler}
interface, you can just use the object's ability to parse command-line options
via the \texttt{setOptions(String[])} method (the counterpart of this method is
\texttt{getOptions()}, which returns a \texttt{String[]} array). The
difference between the two approaches is, that the \texttt{setOptions(String[])}
method cannot be used to set the options incrementally. Default values are used
for all options that haven't been explicitly specified in the options array.

The most basic approach is to assemble the String array by hand. The following
example creates an array with a single option (``\texttt{-R}'') that takes an
argument (``\texttt{1}''):
\begin{verbatim}
  String[] options = new String[2];
  options[0] = "-R";
  options[1] = "1";
\end{verbatim}
Since the \texttt{setOptions(String[])} method expects a fully parsed and
correctly split up array (which is done by the command-line), some common
pitfalls with this approach are:
\begin{tight_itemize}
	\item Combination of option and argument -- Using ``\texttt{-R 1}'' as an
element of the String array will fail, prompting WEKA to output an error message
stating that the option ``R 1'' is unknown.
	\item Trailing blanks -- Using ``\texttt{-R }'' will fail as well, since no
removal of trailing blanks happens and therefore option ``R '' is unknown.
\end{tight_itemize}
The easiest way to avoid these problems, is to provide a String array that has
been generated automatically from a single command-line string using the
\texttt{splitOptions(String)} method of the \texttt{weka.core.Utils} class.
Here is an example:
\begin{verbatim}
  import weka.core.Utils;
  ...
  String[] options = Utils.splitOptions("-R 1");
\end{verbatim}
As this method ignores whitespaces, using ``\texttt{  -R    1}'' or
``\texttt{-R 1 }'' will return the same result as ``\texttt{-R 1}''.

Complicated command-lines with lots of nested options, e.g., options for the
support-vector machine classifier \textit{SMO} (package
\texttt{weka.classifiers.functions}) including a kernel setup, are a bit tricky,
since Java requires one to escape double quotes and backslashes inside a
String. The Wiki\cite{wekawiki} article ``Use Weka in your Java code''
references the Java class \texttt{OptionsToCode}, which turns any command-line
into appropriate Java source code.

\newpage

%%%%%%%%%%%%%%%%
% Loading data %
%%%%%%%%%%%%%%%%
\section{Loading data}
\label{api_loading_data}
Before any filter, classifier or clusterer can be applied, data needs to be
present. WEKA enables one to load data from files (in various file formats) and
also from databases. In the latter case, it is assumed in that the database
connection is set up and working. See chapter \ref{databases} for more details
on how to configure WEKA correctly and also more information on JDBC (Java
Database Connectivity) URLs.

The following classes are used to store data in memory:
\begin{tight_itemize}
	\item \texttt{weka.core.Instances} -- holds a complete dataset. This data
structure is row-based, single rows can be accessed via the
\texttt{instance(int)} method using a 0-based index. Information about the
columns can be accessed via the \texttt{attribute(int)} method. This method
returns \texttt{weka.core.Attribute} objects (see below).
	\item \texttt{weka.core.Instance} -- encapsulates a single row. It is
basically a wrapper around an array of double primitives. Since this class
contains no information about the type of the columns, it always needs access
to a \texttt{weka.core.Instances} object (see methods \texttt{dataset} and
\texttt{setDataset}). The class \texttt{weka.core.SparseInstance} is used in
case of sparse data.
	\item \texttt{weka.core.Attribute} -- holds the information about a single
column in the dataset. It stores the type of the attribute, as well as the
labels for \textit{nominal} attributes, the possible values for \textit{string}
attributes or the datasets for \textit{relational} attributes (these are just
\texttt{weka.core.Instances} objects again).
\end{tight_itemize}

\subsection{Loading data from files}
When loading data from files, one can either let WEKA choose the appropriate
loader (the available loaders can be found in the \texttt{weka.core.converters}
package) based on the file's extension or one can use the correct loader
explicitly. The latter case is necessary if the files don't have the correct
extension.

The \texttt{DataSource} class (inner class of the
\texttt{weka.core.converters.ConverterUtils} class) can be used to read data
from files that have the appropriate file extension. Here are a some examples:
\begin{verbatim}
   import weka.core.converters.ConverterUtils.DataSource;
   import weka.core.Instances;
   ...
   Instances data1 = DataSource.read("/some/where/dataset.arff");
   Instances data2 = DataSource.read("/other/place/dataset.csv");
   Instances data3 = DataSource.read("/other/place/dataset.xrff");
\end{verbatim}
In case the file does have a different file extension than is normally
associated with the loader, one has to use a loader directly. The following
example loads a CSV file (``comma-separated values'') file:
\begin{verbatim}
   import weka.core.converters.CSVLoader;
   import weka.core.Instances;
   import java.io.File;
   ...
   CSVLoader loader = new CSVLoader();
   loader.setSource(new File("/some/where/some.data"));
   Instances data = loader.getDataSet();
\end{verbatim}
\textbf{NB:} Not all file formats allow to store information about the class
attribute (e.g., ARFF stores no information about class attribute, but XRFF
does). If a class attribute is required further down the road, e.g., when using
a classifier, it can be set with the \texttt{setClassIndex(int)} method:
\begin{verbatim}
   // uses the first attribute as class attribute
   if (data.classIndex() == -1)
      data.setClassIndex(0);
   ...
   // uses the last attribute as class attribute
   if (data.classIndex() == -1)
      data.setClassIndex(data.numAttributes() - 1);
\end{verbatim}

\subsection{Loading data from databases}
For loading data from databases, you have two classes that allow you to do that:
\begin{tight_itemize}
	\item \texttt{weka.experiment.InstanceQuery}
	\item \texttt{weka.core.converters.DatabaseLoader}
\end{tight_itemize}
The differences between them are, that the \texttt{InstanceQuery} class allows
you to retrieve sparse data and the \texttt{DatabaseLoader} can retrieve the
data incrementally. \\

\noindent Here is an example of using the \texttt{InstanceQuery} class:
\begin{verbatim}
  import weka.core.Instances;
  import weka.experiment.InstanceQuery;
  ...
  InstanceQuery query = new InstanceQuery();
  query.setDatabaseURL("jdbc_url");
  query.setUsername("the_user");
  query.setPassword("the_password");
  query.setQuery("select * from whatsoever");
  // if your data is sparse, then you can say so too
  // query.setSparseData(true);
  Instances data = query.retrieveInstances();
\end{verbatim}
And an example using the \texttt{DatabaseLoader} class in ``batch retrieval'':
\begin{verbatim}
   import weka.core.Instances;
   import weka.core.converters.DatabaseLoader;
   ...
   DatabaseLoader loader = new DatabaseLoader();
   loader.setSource("jdbc_url", "the_user", "the_password");
   loader.setQuery("select * from whatsoever");
   Instances data = loader.getDataSet();
\end{verbatim}

\samepage
\noindent And in ``incremental mode'':
\begin{verbatim}
   import weka.core.Instance;
   import weka.core.Instances;
   import weka.core.converters.DatabaseLoader;
   ...
   DatabaseLoader loader = new DatabaseLoader();
   loader.setSource("jdbc_url", "the_user", "the_password");
   loader.setQuery("select * from whatsoever");
   Instances structure = loader.getStructure();
   Instances data = new Instances(structure);
   Instance inst;
   while ((inst = loader.getNextInstance(structure)) != null)
      data.add(inst);
\end{verbatim}
\textbf{Notes:}
\begin{tight_itemize}
	\item Not all database systems allow incremental retrieval.
	\item Not all queries have a unique key to retrieve rows incrementally. In
that case, you can supply the necessary columns with the
\textit{setKeys(String)} method (comma-separated list of columns).
	\item If the data cannot be retrieved in an incremental fashion, it is first
loaded fully into memory and then provided row-by-row (``pseudo-incremental'').
\end{tight_itemize}

\newpage

%%%%%%%%%%%%%
% Filtering %
%%%%%%%%%%%%%
\section{Filtering}
In WEKA, filters are used to preprocess the data. They can be found in
package \texttt{weka.filters}. Each filter falls into one
of the following two categories:
\begin{tight_itemize}
	\item \textit{supervised} -- The filters require a class attribute to be
set.
	\item \textit{unsupervised} -- A class attribute is not required to be
present.
\end{tight_itemize}
And into one of the two sub-categories:
\begin{tight_itemize}
	\item \textit{attribute-based} -- Columns are processed, e.g., added or
removed .
	\item \textit{instance-based} -- Rows are processed, e.g., added or deleted.
\end{tight_itemize}
These categories should make it clear, what the difference between the two
\texttt{Discretize} filters in WEKA is. The \textit{supervised} one takes the
class attribute and its distribution over the dataset into account, in order to
determine the optimal number and size of bins, whereas the
\textit{unsupervised} one relies on a user-specified number of bins.

Apart from this classification, filters are either \textit{stream-} or
\textit{batch-based}. \textit{Stream} filters can process the data straight away
and make it immediately available for collection again. \textit{Batch} filters,
on the other hand, need a batch of data to setup their internal
data structures. The \texttt{Add} filter (the can be found in the
\texttt{weka.filters.unsupervised.attribute} package) is an example of a stream
filter. Adding a new attribute with only missing values does not require any
sophisticated setup. However, the \texttt{ReplaceMissingValues} filter (same
package as the \texttt{Add} filter) needs a batch of data in order to determine
the means and modes for each of the attributes. Otherwise the filter will not
be able to replace the missing values with meaningful values. But as soon as a
batch filter has been initialized with the first batch of data, it can also
process data on a row-by-row basis, just like a stream filter.

\textit{Instance-based} filters are a bit special in the way they handle data.
As mentioned earlier, \textit{all} filters can process data on a row-by-row
basis after the first batch of data has been passed through. Of course, if a
filter adds or removes rows from a batch of data, this no longer works when
working in single-row processing mode. It makes sense, if one thinks of a
scenario involving the \texttt{FilteredClassifier} meta-classifier: after
the training phase (=\ first batch of data), the classifier will get
evaluated against a test set, one instance at a time. If the filter now
removes the only instance or adds instances, it can no longer be evaluated
correctly, as the evaluation expects to get only a single result back. This is
the reason why \textit{instance-based} filters only pass through any subsequent
batch of data without processing it. The \texttt{Resample} filters, for
instance, act like this.

\newpage

The following example uses the \texttt{Remove} filter (filter can be
found in package \texttt{weka.filters.unsupervised.attribute}) to remove the
first attribute from a dataset. For setting the options, the
\texttt{setOptions(String[])} method is used.
\begin{verbatim}
   import weka.core.Instances;
   import weka.filters.Filter;
   import weka.filters.unsupervised.attribute.Remove;
   ...
   String[] options = new String[2];
   options[0] = "-R";                   // "range"
   options[1] = "1";                    // first attribute
   Remove remove = new Remove();        // new instance of filter
   remove.setOptions(options);          // set options
   remove.setInputFormat(data);         // inform filter about dataset
                                        // **AFTER** setting options
   Instances newData = Filter.useFilter(data, remove); // apply filter
\end{verbatim}
A common trap to fall into is setting options after the
\texttt{setInputFormat(Instances)} has been called already. Since this method
is used to determine the output format of the data, \textbf{all} the options
have to be set before calling it. Otherwise, all options set afterwards will be
ignored.

\subsection{Batch filtering}
Batch filtering is necessary if two or more datasets need to processed
according to the same filter initialization. For instance, standardizing a
training and a test set with the \texttt{Standardize} filter (package
\texttt{weka.filters.unsupervised.attribute}). The example below shows how to
do this:
\begin{verbatim}
   Instances train = ...   // from somewhere
   Instances test = ...    // from somewhere
   Standardize filter = new Standardize();
   // initializing the filter once with training set
   filter.setInputFormat(train);
   // configures the Filter based on train instances and returns
   // filtered instances
   Instances newTrain = Filter.useFilter(train, filter);
   // create new test set
   Instances newTest = Filter.useFilter(test, filter);
\end{verbatim}

\subsection{Filtering on-the-fly}
Even though using the API gives one full control over the data and makes it
easier to juggle several datasets at the same time, filtering data
\textbf{on-the-fly} makes life even easier. This handy feature is available
through meta schemes in WEKA, like \texttt{FilteredClassifier} (package
\texttt{weka.classifiers.meta}), \texttt{FilteredClusterer} (package
\texttt{weka.clusterers}), \texttt{FilteredAssociator} (package
\texttt{weka.associations}) and
\texttt{FilteredAttributeEval}/\texttt{FilteredSubsetEval} (package
\texttt{weka.attributeSelection}). Instead of filtering the data
\textit{beforehand}, one just sets up a meta-scheme and lets the meta-scheme
do the filtering for one.

\newpage

The following example uses the
\texttt{FilteredClassifier} in conjunction with the \texttt{Remove} filter to
remove the first attribute from the dataset and \texttt{J48} (package
\texttt{weka.classifiers.trees}) as base-classifier. First the classifier is
built with a training set and the evaluated with a separate test set. The
actual and predicted class values are printed in the console. For more
information on classification, see chapter \ref{api_classiyfing}.
\begin{verbatim}
   import weka.classifiers.meta.FilteredClassifier;
   import weka.classifiers.trees.J48;
   import weka.filters.unsupervised.attribute.Remove;
   ...
   Instances train = ...         // from somewhere
   Instances test = ...          // from somewhere
   // filter
   Remove rm = new Remove();
   rm.setAttributeIndices("1");  // remove 1st attribute
   // classifier
   J48 j48 = new J48();
   j48.setUnpruned(true);        // using an unpruned J48
   // meta-classifier
   FilteredClassifier fc = new FilteredClassifier();
   fc.setFilter(rm);
   fc.setClassifier(j48);
   // train and make predictions
   fc.buildClassifier(train);
   for (int i = 0; i < test.numInstances(); i++) {
     double pred = fc.classifyInstance(test.instance(i));
     double actual = test.instance(i).classValue();
     System.out.print("ID: "
       + test.instance(i).value(0));
     System.out.print(", actual: "
       + test.classAttribute().value((int) actual));
     System.out.println(", predicted: "
       + test.classAttribute().value((int) pred));
   }
\end{verbatim}

\newpage

%%%%%%%%%%%%%%%
% Classifying %
%%%%%%%%%%%%%%%
\section{Classifying}
\label{api_classiyfing}
Classification and regression algorithms in WEKA are called ``classifiers'' and
are located in the \texttt{weka.classifiers} package. This section covers the
following topics:
\begin{tight_itemize}
	\item \textit{Building a classifier} -- batch and incremental learning.
	\item \textit{Evaluating a classifier} -- various evaluation techniques and
how to obtain the generated statistics.
	\item \textit{Classifying instances} -- obtaining classifications for
unknown data.
\end{tight_itemize}

\subsection{Building a classifier}
By default, all classifiers in WEKA are \textit{batch-trainable}, i.e., the
train on the whole dataset at once. This is fine, if the training data fits
into memory. But there are also algorithms available, that can update their
internal model \textit{on-the-go}. These classifiers are called
\textit{incremental}. The following two sections cover the batch and the
incremental classifiers.

\subsubsection*{Batch classifiers}
A batch classifier is really simple to build:
\begin{tight_itemize}
	\item set options -- either using the \texttt{setOptions(String[])} method
or the actual set-methods.
	\item train it -- calling the \texttt{buildClassifier(Instances)} method
with the training set. By definition, the \texttt{buildClassifier(Instances)}
method resets the internal model completely, in order to ensure that subsequent
calls of this method with the same data result in the same model.
\end{tight_itemize}
The following code snippet builds an unpruned J48 (J48 is WEKA's C4.5
implementation) on a dataset:
\begin{verbatim}
   import weka.core.Instances;
   import weka.classifiers.trees.J48;
   ...
   Instances data = ...              // from somewhere
   String[] options = new String[1];
   options[0] = "-U";                // unpruned tree
   J48 tree = new J48();             // new instance of tree
   tree.setOptions(options);         // set the options
   tree.buildClassifier(data);       // build classifier
\end{verbatim}

\subsubsection*{Incremental classifiers}
All incremental classifiers in WEKA implement the interface
\texttt{UpdateableClassifier} (found in package \texttt{weka.classifiers}).
Bringing up the Javadoc for this particular interface tells one what classifiers
implement this interface. Those classifiers can be used to process large amounts
of data with a small memory-footprint, as the training data does not have to fit
in memory. ARFF files, for instance, can be read incrementally (see chapter
\ref{api_loading_data}).

Training an incremental classifier happens in \textit{two} stages:
\begin{tight_enumerate}
	\item \textit{initialize} the model by calling the
\texttt{buildClassifier(Instances)} method. One can either use an
\texttt{weka.core.Instances} object with no actual data or one with an initial
set of data.
	\item \textit{update} the model row-by-row by calling the
\texttt{updateClassifier(Instance)} method.
\end{tight_enumerate}
The following example shows how load an ARFF file incrementally using the
\texttt{ArffLoader} class, training the
\texttt{NaiveBayesUpdateable} classifier.
\begin{verbatim}
   import weka.core.converters.ArffLoader;
   import weka.classifiers.bayes.NaiveBayesUpdateable;
   import java.io.File;
   ...
   // load data
   ArffLoader loader = new ArffLoader();
   loader.setFile(new File("/some/where/data.arff"));
   Instances structure = loader.getStructure();
   structure.setClassIndex(structure.numAttributes() - 1);

   // train NaiveBayes
   NaiveBayesUpdateable nb = new NaiveBayesUpdateable();
   nb.buildClassifier(structure);
   Instance current;
   while ((current = loader.getNextInstance(structure)) != null)
     nb.updateClassifier(current);
\end{verbatim}

\newpage

\subsection{Evaluating a classifier}
Building a classifier is only one part of the equation, evaluating how
\textit{well} it performs is another important part. WEKA supports two types of
evaluation:
\begin{tight_itemize}
	\item \textit{Cross-validation} -- If you only have a single dataset and
want to get a reasonable realistic evaluation. Setting the number of folds equal
to the number of rows in the dataset will give you leave-one-out
cross-validation (LOOCV).
	\item \textit{Dedicated test set} -- The test set is solely used to evaluate
the built classifier. It is important to have a test set that incorporates the
same (or similar) concepts as the training set, otherwise you will always end
up with poor performance.
\end{tight_itemize}
The evaluation step, and collection of statistics, is performed by the
\texttt{Evaluation} class (package \texttt{weka.classifiers}).

\subsubsection*{Cross-validation}
The \texttt{crossValidateModel} method of the \texttt{Evaluation} class is used
to perform cross-validation with an \textbf{untrained} classifier and a single
dataset. Supplying an untrained classifier ensures that no information leaks
into the actual evaluation. Even though it is an implementation requirement,
that the \texttt{buildClassifier} method resets the classifier, it cannot be
guaranteed that this is indeed the case (``leaky'' implementation). Using an
untrained classifier avoids unwanted side-effects, as for each train/test set
pair, a copy of the originally supplied classifier is used.

Before cross-validation is performed, the data gets randomized using the
supplied random number generator (\texttt{java.util.Random}). It is
recommended that this number generator is ``seeded'' with a specified seed
value. Otherwise, subsequent runs of cross-validation on the same dataset will
not yield the same results, due to different randomization of the data.

The code snippet below performs 10-fold cross-validation with a J48 decision
tree algorithm on a dataset \texttt{newData}, with random number generator that
is seeded with ``1''. The summary of the collected statistics is output to
\texttt{stdout}.
\begin{verbatim}
   import weka.classifiers.Evaluation;
   import weka.classifiers.trees.J48;
   import java.util.Random;
   ...
   Instances newData = ... // from somewhere
   Evaluation eval = new Evaluation(newData);
   J48 tree = new J48();
   eval.crossValidateModel(tree, newData, 10, new Random(1));
   System.out.println(eval.toSummaryString("\nResults\n\n", false));
\end{verbatim}
The \texttt{Evaluation} object in this example is initialized with the dataset
used in the evaluation process. This is done in order to inform the evaluation
about the type of data that is being evaluated, ensuring that all internal data
structures are setup correctly.

In order to get a more dataset-dependent initial randomization of the data, the
\texttt{getRandomNumberGenerator(int)} method of the
\texttt{weka.core.Instances} class can be used for this purpose. This method
returns a \texttt{java.util.Random} object that was seeded with the sum
of the supplied seed and the hashcode of the string representation of
a randomly chosen \texttt{weka.core.Instance} of the \texttt{Instances} object
(using a random number generator seeded with the seed supplied to this method).

\subsubsection*{Train/test set}
Using a dedicated test set to evaluate a classifier is just as easy as
cross-validation. But instead of providing an untrained classifier, a trained
classifier has to be provided now. Once again, the
\texttt{weka.classifiers.Evaluation} class is used to perform the evaluation,
using the \texttt{evaluateModel} method.

The code snippet below trains a J48 with default options on a training set and
evaluates on a test set before outputting the summary of the
collected statistics:
\begin{verbatim}
   import weka.core.Instances;
   import weka.classifiers.Evaluation;
   import weka.classifiers.trees.J48;
   ...
   Instances train = ...   // from somewhere
   Instances test = ...    // from somewhere
   // train classifier
   Classifier cls = new J48();
   cls.buildClassifier(train);
   // evaluate classifier and print some statistics
   Evaluation eval = new Evaluation(train);
   eval.evaluateModel(cls, test);
   System.out.println(eval.toSummaryString("\nResults\n\n", false));
\end{verbatim}

\newpage

\subsubsection*{Statistics}
In the previous sections, the \texttt{toSummaryString} of the
\texttt{Evaluation} class was already used in the code examples. But there are
other summary methods for nominal class attributes available as well:
\begin{tight_itemize}
	\item \texttt{toMatrixString}
	\item \texttt{toClassDetailsString}
	\item \texttt{toCumulativeMarginDistributionString}
\end{tight_itemize}
If one does not want to use these summary methods, it is possible to access the
individual statistical measures directly. Below, a few common measures
are listed:
\begin{tight_itemize}
	\item nominal class attribute
	\begin{tight_itemize}
		\item \texttt{correct()} -- The number of correctly classified
instances. The incorrectly classified ones are available through
\texttt{incorrect()}.
		\item \texttt{pctCorrect()} -- The percentage of correctly classified
instances (accuracy). \texttt{pctIncorrect()} returns the number of
misclassified ones.
		\item \texttt{areaUnderROC(int)} -- The AUC for the specified class
label index (0-based index).
	\end{tight_itemize}

	\item numeric class attribute
	\begin{tight_itemize}
		\item \texttt{correlationCoefficient()} -- The correlation coefficient.
	\end{tight_itemize}

	\item general
	\begin{tight_itemize}
		\item \texttt{meanAbsoluteError()} -- The mean absolute error.
		\item \texttt{rootMeanSquaredError()} -- The root mean squared error.
		\item \texttt{numInstances()} -- The number of instances with a class
value.
		\item unclassified() - The number of unclassified instances.
		\item pctUnclassified() - The percentage of unclassified instances.
	\end{tight_itemize}
\end{tight_itemize}
For a complete overview, see the Javadoc page of the \texttt{Evaluation} class.
By looking up the source code of the summary methods mentioned above, you can
easily determine what methods are used for which particular output.

\newpage

\subsubsection*{ROC curves}
WEKA can generate Receiver operating characteristic (ROC) curves, based on the
collected predictions during an evaluation. The class used for generating the
necessary data to plot is \texttt{ThresholdCurve} (package
\texttt{weka.classifiers.evaluation}). The \texttt{getCurve} method of
this class generates a \texttt{weka.core.Instances} object for a particular
class label (specified with a 0-based index), based on the collected predictions
returned from an \texttt{Evaluation} object's \texttt{predictions()} method.

\noindent The following code snippet generates the plottable data
(\texttt{result}) for the first class label:
\begin{verbatim}
   import weka.classifiers.Evaluation;
   import weka.classifiers.evaluation.ThresholdCurve;
   ...
   Evaluation eval = ... // from somewhere
   ThresholdCurve tc = new ThresholdCurve();
   int classIndex = 0;
   Instances result = tc.getCurve(eval.predictions(), classIndex);
\end{verbatim}
Using the \texttt{ThresholdVisualizePanel} class (package
\texttt{weka.gui.visualize}), the plottable data stored in \texttt{result} can
then be displayed in a frame. The \texttt{ThresholdCurve} can be used to
determine the AUC, using the \texttt{getROCArea(Instances)} method.
\begin{verbatim}
   import weka.classifiers.evaluation.ThresholdCurve;
   import weka.core.Utils;
   import weka.gui.visualize.PlotData2D;
   import weka.gui.visualize.ThresholdVisualizePanel;
   ...
   ThresholdVisualizePanel vmc = new ThresholdVisualizePanel();
   vmc.setROCString("(Area under ROC = " +
       Utils.doubleToString(ThresholdCurve.getROCArea(result), 4) + ")");
   vmc.setName(result.relationName());
   PlotData2D tempd = new PlotData2D(result);
   tempd.setPlotName(result.relationName());
   tempd.addInstanceNumberAttribute();
   vmc.addPlot(tempd);
   String plotName = vmc.getName();
   final javax.swing.JFrame jf =
     new javax.swing.JFrame("Weka Classifier Visualize: "+plotName);
   jf.setSize(500,400);
   jf.getContentPane().setLayout(new BorderLayout());
   jf.getContentPane().add(vmc, BorderLayout.CENTER);
   jf.addWindowListener(new java.awt.event.WindowAdapter() {
     public void windowClosing(java.awt.event.WindowEvent e) {
       jf.dispose();
     }
   });
   jf.setVisible(true);
\end{verbatim}

\newpage

\subsection{Classifying instances}

\newpage

%%%%%%%%%%%%%%
% Clustering %
%%%%%%%%%%%%%%
\section{Clustering}

\newpage

%%%%%%%%%%%%%%%%%%%%%%%%
% Selecting attributes %
%%%%%%%%%%%%%%%%%%%%%%%%
\section{Selecting attributes}

\newpage

%%%%%%%%%%%%%%%
% Saving data %
%%%%%%%%%%%%%%%
\section{Saving data}
Saving \texttt{weka.core.Instances} objects is as easy as reading the data in
the first place, though the process of storing the data again is far less common
than of reading the data into memory. The following two sections cover how to
save the data in files and in databases.

\subsection{Saving data to files}
Once again, one can either let WEKA choose the appropriate converter for saving
the data or use an explicit converter (all savers are located in the
\texttt{weka.core.converters} package). The latter approach is necessary, if the
file name under which the data will be stored does not have an extension that
WEKA recognizes.

Use the \texttt{DataSink} class (inner class of
\texttt{weka.core.converters.ConverterUtils}), if the extensions are not a
problem. Here are a few examples:
\begin{verbatim}
   import weka.core.Instances;
   import weka.core.converters.ConverterUtils.DataSink;
   ...
   // data structure to save
   Instances data = ...
   // save as ARFF
   DataSink.write("/some/where/data.arff");
   // save as CSV
   DataSink.write("/some/where/data.csv");
\end{verbatim}
And here is an example of using the \texttt{CSVSaver} converter explicitly:
\begin{verbatim}
   import weka.core.Instances;
   import weka.core.converters.CSVSaver;
   import java.io.File;
   ...
   // data structure to save
   Instances data = ...
   // save as CSV
   CSVSaver saver = new CSVSaver();
   saver.setInstances(data);
   saver.setFile(new File("/some/where/data.csv"));
   saver.writeBatch();
\end{verbatim}

\subsection{Saving data to databases}
Apart from the KnowledgeFlow, saving to databases is not very obvious in WEKA,
unless you know about the \texttt{DatabaseSaver} converter. Just like the
\texttt{DatabaseLoader}, the saver counterpart can store the data either in
batch mode or incrementally as well.

The first example shows how to save the data in batch mode, which is the easier
way of doing it:
\begin{verbatim}
   import weka.core.Instances;
   import weka.core.converters.DatabaseSaver;
   ...
   // data structure to save
   Instances data = ...
   // store data in database
   DatabaseSaver saver = new DatabaseSaver();
   saver.setDestination("jdbc_url", "the_user", "the_password");
   // we explicitly specify the table name here:
   saver.setTableName("whatsoever2");
   saver.setRelationForTableName(false);
   // or we could just update the name of the dataset:
   // saver.setRelationForTableName(true);
   // data.setRelationName("whatsoever2");
   saver.setInstances(data);
   saver.writeBatch();
\end{verbatim}
Saving the data incrementally, requires a bit more work, as one has to specify
that writing the data is done incrementally (using the \texttt{setRetrieve}
method), as well as notifying the saver when all the data has been saved:
\begin{verbatim}
   import weka.core.Instances;
   import weka.core.converters.DatabaseSaver;
   ...
   // data structure to save
   Instances data = ...
   // store data in database
   DatabaseSaver saver = new DatabaseSaver();
   saver.setDestination("jdbc_url", "the_user", "the_password");
   // we explicitly specify the table name here:
   saver.setTableName("whatsoever2");
   saver.setRelationForTableName(false);
   // or we could just update the name of the dataset:
   // saver.setRelationForTableName(true);
   // data.setRelationName("whatsoever2");
   saver.setRetrieval(DatabaseSaver.INCREMENTAL);
   saver.setStructure(data);
   count = 0;
   for (int i = 0; i < data.numInstances(); i++) {
     saver.writeIncremental(data.instance(i));
   }
   // notify saver that we're finished
   saver.writeIncremental(null);
\end{verbatim}
